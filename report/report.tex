\documentclass[letterpaper]{article}
\usepackage{aaai}
\usepackage{times}
\usepackage{helvet}
\usepackage{courier}
\RequirePackage{booktabs}
\setlength{\pdfpagewidth}{8.5in} 
\setlength{\pdfpageheight}{11in}


\setcounter{secnumdepth}{0}

\title{AI project}
\author{Rudra Sharma \and Joshua O'Dell \\ Colorado State University \\ Fort Collins, CO 80523 }
\begin{document}
\maketitle

\section{Introduction}

The purpose of this project is to simulate an automated vacuum cleaner (AUC). We will represent a room as a 2D plane which has clean and dirty points/tiles. There will also be obstacles in the room such as furniture which would obstruct paths for a vacuum. The goal of the vacuum cleaner is to clear all the dirty points/tiles and navigate/find a path around the obstacles in the room.

In the project we will implement a variety of different vacuums and study its impact on performance. There have been several papers written on the study of automated vacuum cleaners. IRobot’s Roomba is an example of such a robot, we will study this implementation when developing our own algorithms.

\section{Experiment Implementation}

The project implemented 6 distinct vacuum robots. Each Robot has a different set of percepts, and the algorithm is adjusted to deal with it.

The following is a list of the capabilities of each Vacuum robot.
\begin{enumerate}  
\item \textbf{Preloaded Map:} Before the cleaning begins a map of the world is loaded into each the vacuum robot.
\item \textbf{Store explored nodes:} As the vacuum robot moves through the world it is allowed to store with infinite storage the tiles where it has been and the status of that tile
\item \textbf{Dirt Sensor:} A vacuum is allowed to sense the status of the tile that it is currently on.
% are we going to allow it to sense neighbors?
\item \textbf{Proximity Sensor:} the vacuum robot has the ability to know when it is close to a wall.
% assume direct neighbor, but possibly farther?
\end{enumerate}  


\subsection{Program 1}

\begin{tabular}{ r | l }  
	Preloaded Map			& Yes \\
	Store explored nodes	& Yes \\
	Dirt Sensor				& Yes \\
	Proximity Sensor		& No \\
\end{tabular}

This program was implemented by making an algorithm...

We discovered that this program did something...


\subsection{Program 2}

\begin{tabular}{ r | l }  
	Preloaded Map			& Yes \\
	Store explored nodes	& No \\
	Dirt Sensor				& Yes \\
	Proximity Sensor		& No \\
\end{tabular}

This program was implemented by making an algorithm...

We discovered that this program did something...


\subsection{Program 3}

\begin{tabular}{ r | l }  
	Preloaded Map			& Yes \\
	Store explored nodes	& Yes \\
	Dirt Sensor				& No \\
	Proximity Sensor		& Yes \\
\end{tabular}

This program was implemented by making an algorithm...

We discovered that this program did something...


\subsection{Program 4}

\begin{tabular}{ r | l }  
	Preloaded Map			& No \\
	Store explored nodes	& No \\
	Dirt Sensor				& Yes \\
	Proximity Sensor		& Yes \\
\end{tabular}

This program was implemented by making an algorithm...

We discovered that this program did something...


\subsection{Program 5}

\begin{tabular}{ r | l }  
	Preloaded Map			& Yes \\
	Store explored nodes	& Yes \\
	Dirt Sensor				& No \\
	Proximity Sensor		& No \\
\end{tabular}

This program was implemented by making an algorithm...

We discovered that this program did something...


\subsection{Program 6}

\begin{tabular}{ r | l }  
	Preloaded Map			& No \\
	Store explored nodes	& No \\
	Dirt Sensor				& Yes \\
	Proximity Sensor		& No \\
\end{tabular}

This program was implemented by making an algorithm...

We discovered that this program did something...


\section{Results}

We found the following.

\begin{tabular}{ r | l }  
	\toprule
		& avg \% cleaned \\
	\midrule
	P1	& 15 \\
	P2	& 45 \\
	P3	& 84 \\
	P4	& 56 \\
	P5	& 77 \\
	P6	& 1 \\
	\bottomrule
\end{tabular}

\section{Conclusion}

% \cite[p. 2] {Optimized_Path_Planning}

\nocite{*}
\bibliography{report}
\bibliographystyle{aaai}
\end{document}


